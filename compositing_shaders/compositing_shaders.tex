% "Compositing shaders in X3D" by Michalis Kamburelis,
% for Web3D 2011.
%
% If accepted, this will be © Copyright 2010 by ACM, Inc.
% See http://www.acm.org/publications/policies/copyright_policy
%

%% Template stuff begins here ------------------------------------------------

\documentclass{acmsiggraph}                     % final
%\documentclass[annualconference]{acmsiggraph}  % final (annual conference)
%\documentclass[review]{acmsiggraph}            % review
%\documentclass[widereview]{acmsiggraph}        % wide-spaced review
%\documentclass[preprint]{acmsiggraph}          % preprint

\usepackage[scaled=.92]{helvet}
\usepackage{times}
\usepackage{graphicx}
\usepackage{parskip}
\usepackage[labelfont=bf,textfont=it]{caption}
\onlineid{1010} %% I think this doesn't matter

%% Template stuff ends here ------------------------------------------------

% This is really the only sensible way to make breaking of monospace text
% (everything inside \texttt, including (but not limited) to urls).
% Otherwise, the monospace text flows outside of the column all over the place.
\sloppy

\usepackage{needspace}

%% For href
\usepackage{ifpdf}
\ifpdf
  \usepackage[pdftex]{hyperref}
\else
  \usepackage[hypertex]{hyperref}
\fi

%% Put float in a nice box,
%% http://en.wikibooks.org/wiki/LaTeX/Floats,_Figures_and_Captions
\usepackage{float}
\floatstyle{boxed}
\newfloat{mycodecore}{H}{listofmycode}
\floatname{mycodecore}{}

% Fix mycodecore: it has additional vertical line at the bottom after the frame.
% Possibly due to Verbatim inside?
\newenvironment{mycode}
{\begin{mycodecore}}
{\end{mycodecore}
\vspace{-0.1in}}

%% Use verbatim that allows \latex commands inside,
%% highly useful for my node spec figures.
%% See http://scott.sherrillmix.com/blog/category/programmer/latex/,
%% http://www.ctan.org/tex-archive/macros/latex/contrib/fancyvrb/
\usepackage{fancyvrb}

%% To allow tex_projected at the bottom.
%% Without this, figure* can only go to the top (or separate page).
%% http://en.wikibooks.org/wiki/LaTeX/Floats,_Figures_and_Captions#Wide_figures_in_two_column_documents
\usepackage{stfloats}

%% Bold inside our code/spec samples.
%%
%% For an older acm style (from
%% http://www.acm.org/sigs/publications/proceedings-templates)
%% this was a hack:
%% we get some strange teletype font that cannot be bold (it just looks
%% the same). It's a pity, since my verbatims really depend on bold parts
%% for good look. Workaround was to just resign from teletype.
%%
%% For acm siggraph from http://www.siggraph.org/publications/instructions/index.html
%% this simply makes text bold, no tricks.
\newcommand*{\codeem}[1]{\textbf{#1}}

% \nolinkurl is a trick to allow links to break across lines in pdf,
% from http://www.miwie.org/tex-refs/html/latex-packages.html#hyperref
\newcommand*{\myhref}[2]{\texttt{\href{#1}{\nolinkurl{#2}}}}

% The above \myhref unfortunately (due to \nolinkurl trick)
% makes it impossible to break links manually.
% So below is the same as \myhref, but does not break link text automatically,
% and allows to break it manually (by inserting \\ etc.)
% \newcommand*{\myhrefm}[2]{\texttt{\href{#1}{#2}}}

\newenvironment{myenumerate}
{\begin{enumerate}
  \setlength{\itemsep}{0pt}
  \setlength{\parskip}{0pt}
  \setlength{\parsep}{0pt}}
{\end{enumerate}}

\title{Compositing shaders in X3D}

\author{Michalis Kamburelis\thanks{e-mail: michalis.kambi@gmail.com}\\Institute of Computer Science\\University of Wroc{\l}aw, Poland}
%% no full addr here, I think?
%% ul. Joliot-Curie 15
%% 50-383 Wroc{\l}aw, Poland

%% TODO: screenshots

\keywords{X3D graphics, shaders, GLSL, shadows, shadow maps, bump mapping}

\begin{document}

\maketitle

\begin{abstract}
TODO
\end{abstract}

\begin{CRcatlist}
%% TODO: update it
  \CRcat{I.3.7}{Computer Graphics}{Three-Dimensional Graphics and Realism}{Color, shading, shadowing, and texture};
  \CRcat{I.3.6}{Computer Graphics}{Methodology and Techniques}{Languages, Standards}
\end{CRcatlist}

\keywordlist

%% \terms{Languages, Standardization, Algorithms}
%% not needed with acmsiggraph template?

\section{Introduction}

%% The ``\copyrightspace'' command must be the first command after the
%% start of the first section of the body of your paper. It ensures the
%% copyright space is left at the bottom of the first column on the first
%% page of your paper.
\copyrightspace

We present a technique for composing many shader snippets into a
single shader code. X3D browsers can use it to allow user and browser
shaders to coexist. Thus, users benefit from all the effects
implemented in the browser, even when they want to add some effect by
explicit shader code. Users can also seamlessly connect various
effects into a shape, without the need to adjust any shader source
code. Even the browser's implementation can be simplified, by using
the same mechanism to separate implementation of various effects
within a browser.

For an interesting example, condider three effects: shadows, bump
mapping, and lighting equations. Shadows provide an information that
either turns on or off a particular lighting contribution (or scales
this contribution, in case of soft shadows). Lighting equation
calculates light contribution, looking at material, light source and
normal at given geometry point. Bump mapping (in the simplest case)
modifies how the normal value is obtained: when bump mapping is used,
the normal is taken from the texture. So we have

$$  shadow(light) * light\_contribution(light, material, normal(point)) $$

In case of no shadows, shadow(x) = 1. In case of no bump mapping,
normal is determined by the geometry (per-face, or smoothed per-face),
and usually is passed as an already calculated attribute to the
shader.

Now, we would like to allow modifying the three functions here to be
orthogonal.

\section{Motivation and previous work}

Current shading description methods do not allow easy connecting
many effects into a single shader.

Shading source code files (GLSL, Cg) make no such way.

Effects formats (HLSL .fx, CgFX) allows to choose one technique, and
within it make passes. No way of automatically connecting one .fx file
with another.

One oldest solution: make a library of functions (like GLSL
functions), and allow to user to use them. But this is very bad and
limited. You cannot e.g. insert a shadow check right before adding the
light contribution (and still use browser code outside (to iterate
over lights) and inside (to calculate single light contribution)).

Another common solution is to arrange shaders in a pipe, where one
shader processes the result of another. This can be visualized as
layers of materials, where each layer modifies the previous
layer. However, this is again not flexible --- there's no way to plug
in your code in the middle of another shader's work.

An older solution, in the spirit of above, was also to perform
multi-pass rendering. However, this (in addition to lack of
flexibility of previous approach) makes also a loss of rendering
speed. In our work, we want to allow a single rendering pass to be as
powerful as it can.

%% (see also http://groups.google.com/group/blendertorenderman/browse_thread/thread/aaf07831b91be9db?pli=1 , confirms my findings)

Another approach is the Sh (ref:http://libsh.org/) language, which
allows writing shaders code (that can run on GPU) directly inside a
C++ program. For this, Sh extends the C++ language (through C++
operator overloading and macros tricks). It allows an excellent
integration between C++ code and shaders, hiding the ugly details of
passing variables between normal code (that executes on CPU) and
shader code (that usually executes on GPU). You can use
object-oriented methods there to create a general shader that can
later be extended through various means, like overriding virtual
methods. However, this is a solution closely couples with C++. It's
suitable if you have a 3D engine in C++, and you want to use in your
own C++ program and extend it's shaders. In this paper, we want to
create a solution that is absolutely separate from the programming
language used to make a browser. Invoking a compiler to generate a
final GPU shader, not to mention teaching user's C++, is out of the
question.

%% ("Metaprogramming GPUs with Sh", http://books.google.com/books?id=8RX4RmFRLmgC&pg=PA79&lpg=PA79&dq=ShAttrib2f&source=bl&ots=IOxd1-BQLL&sig=AIPgFRmbodNpCLWmwk-QV5JUjVo&hl=pl&ei=EylRTe7RB8qeOsiQ1aQI&sa=X&oi=book_result&ct=result&resnum=7&ved=0CEkQ6AEwBg#v=onepage&q=ShAttrib2f&f=false )

\section{Plugging}

The basic idea of our approach is that a shader source code defines
a points when a user-defined code may be inserted. The mechanism
allows very easily to define new plugging points, and to use
them. Moreover, user shaders and browser internal shaders have the
same means to define and use plug points. This allows to connect
shaders (both user and browser) in a variety of ways.

\begin{myenumerate}
\itemsep 0pt
\item Browser shaders can be connected with each other. For example, we
have a basic shader code that defines lighting and texturing. Our bump
mapping is implemented by simply plugging the code in the appropriate
places in this base shader. This way, the base shader is completely
independent from the bump mapping implementatation. We can switch bump
mapping implementations (for example, use bump texture from image or
some random noise) without trouble.

So even inside the engine, this allows for much easier shader generator.

\item User shaders may plug into browser shaders. This is the most usual
case. Our shaders define various points when you can plug your code,
and override or enhance our shading.

\item User shaders may also plug into another user shaders. User shaders
can trivially (by just adding a "magic" comment) define new plugging
points, which are usable by other user shaders. User effects can
define plugging points, which are available to the following effects
on the Apperance.effects list.

\end{myenumerate}

We allow user shaders (defined in ComposedShader and such nodes) to
define additional plugging points. User shaders can use our plug names
(and thus work with the same effects), or they can invent their
own. They can also always add new plug points.

When you don't use an explicit shader (like ComposedShader node),
browser user it's own internal basic shader. This shader is then
extended by effects like bump mapping (coming from internal browser
implementation), or by user effects. But you can also define your own
ComposedShader node. Then your own shader is enhanced by the
browser. TODO: ? If you define the same (or compatible) plugging
points, then the browser effects will be even added to your own
shader. And of course user effects are added to your shader.

In summary, the system works regardless if you use or not an explicit
shader on Appearance.shaders list.

\subsection{ShaderEffect node}

In the X3D file we add a field

\begin{mycode}
\underline{Appearance}
\begin{Verbatim}[commandchars=\\\{\}]
MFNode [] \codeem{effects} [] # ShaderEffect
\end{Verbatim}
\end{mycode}

that can hold any number of

\begin{mycode}
\underline{Effect : X3DNode}
\begin{Verbatim}[commandchars=\\\{\}]
SFString [] \codeem{language} ""
  # just like ComposedShader.language.
  # Will be used only for matching language shaders.
MFString [] \codeem{parts} [] # EffectPart
  ... any number of fields,
      that will be passed
      as uniform to the shader,
      just like for ComposedShader node ...
\end{Verbatim}
\end{mycode}

\begin{mycode}
\underline{EffectPart : X3DNode}
\begin{Verbatim}[commandchars=\\\{\}]
SFString [] \codeem{name} ""
  # plugging point name
MFString [] \codeem{url} ""
  # source code, in external file (url),
  # or inline (following data:text/plain,)
\end{Verbatim}
\end{mycode}

All the effects on this list (with suitable language, and with
plugging points defined by the browser shader, or active shader on
Appearance.shaders list) will be used. Note that this is contrary to
the Appearance.shaders, which chooses only one shader. For effects, we
choose all of them (that are suitable).

The most basic idea of this paper is to allow one to define two
independent shader effects, and then seamlessly connect them by simply
placing them both on Appearance.effects list. This also allows to
define a library of effects, that can be composited without any work
needed by user.

\subsubsection{Defining your own plug points}

When you write shader code, you can define your own plug points by a
magic comment:

/* PLUG: plug-name (param) (const in vec3 foo, inout vec4 bar) */

This defines a point when multiple user plugs may be added. Each
Effect with name matching given plug-name will cause a
definition of a new function, with the declaration as given above
(with \%s replaced by the automatic function name). This function will
be called at given place.

This is often useful for processing some paremeter
repeatedly (like adding or modulating the fragment color),
so there's usually an inout parameter.

When a special value \texttt{declaration} is specified, like

/* PLUG: plug-name declaration */

then the code is inserted directly into the appropriate place.

Multiple code pieces may be added to the given plug point.
So multiple effects may use the same plug point. They are added
in the order they are specified in the Appearance.effects list
(although, preferably, for most effects this order will not matter).

When a special value \texttt{function} is specified, like

/* PLUG: plug-name function type-name (params-called) (params-declared) */

then the code will be replaced to a function call returning given type-name.
Such plug can only be used once.
Identifier (next token in GLSL language) immediately following
this plug will be removed if this plug is defined.

\subsubsection{Why plug as comments}

The nice feature of our magic /* PLUG ... */ comments is that a shader source
is still valid even if you completely ignore them. For example,
you can write your custom ComposedShader node, defining some plugs,
and for browsers that understand them --- the plugs can be used,
for other browsers --- plugs will be gracefully ignored (but still
the shader will run, although without any effects).

\subsubsection{Plugging from other nodes}

The real power of our system comes from the fact that some nodes,
like light sources, textures, or even fog, can define their own effects
now. So you can easily define a shader specific for a given light source,
or texture. For this, we add the \texttt{effects} field to various nodes:

\begin{mycode}
\underline{X3DLightSource, X3DTextureNode, X3DFogObject}
\begin{Verbatim}[commandchars=\\\{\}]
MFNode [] \codeem{effects} [] # ShaderEffect
\end{Verbatim}
\end{mycode}

%% TODO check and fix above X3Dxxx class names.

You can modify the light source contribution, or even replace the default
calculation of light source contribution. In the second case,
the default calculation will not even be used in the shader,
so it will not slow down the calculation without a reason.

% TODO
%% \bibliographystyle{acmsiggraph}
%% \nocite{*}
%% \bibliography{compositing_shaders}

\end{document}
